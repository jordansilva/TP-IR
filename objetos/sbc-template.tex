\documentclass[12pt]{article}

\usepackage{sbc-template}

\usepackage{graphicx,url}
\usepackage[brazil]{babel}   
%\usepackage[latin1]{inputenc}  
\usepackage[T1]{fontenc}
\usepackage[utf8]{inputenc}  
\usepackage{listings}

\newcommand{\tab}[1]{\hspace{.05\textwidth}\rlap{#1}}
% UTF-8 encoding is recommended by ShareLaTex

\sloppy

\title{Programming Assignment \#1\\ User-based Recommendation}

\author{Jordan Silva\inst{1} }

\address{
  Instituto de Ciências Exatas\\
  Universidade Federal de Minas Gerais (UFMG) -- Belo Horizonte, MG -- Brasil
  \email{jordan@dcc.ufmg.br}
}

\begin{document} 

\maketitle


\section{Introdução}

Este trabalho tem como objetivo a implementação e recomendação de filmes utilizando a Filtragem Colaborativa, utilizando a abordagem de recomendação baseada nas preferências e similaridades entre os usuários.
\\
\indent A implementação deste sistema foi realizada utilizando a linguagem \textit{Java 8}; e como entrada de dados, foi utilizado o conjunto de dados \textit{MovieLens 100k Data Set}\cite{movielens100k}.

\section{Filtragem Colaborativa} \label{sec:filtragem}

A técnica de filtragem colaborativa (FC) baseia-se no fato de que as melhores recomendações para um indivíduo podem ser aquelas feitas por pessoas que possuem preferências similares às dele \cite{sampaio2006aprendizagem}. Desta maneira, podemos simplificar que os sistemas que utilizam desta técnica, por natureza, identificam a similaridade entre os usuários e realizam a recomendação baseado nas preferências desses usuários.\\
\indent De acordo com \cite{kajimoto2007sistemas}, na Filtragem Colaborativa o processo de identificação de um item relevante a um usuário é chamado de "\textit{k-nearest-neighbor}" e segue os seguintes critérios: 
 
\begin{itemize}
\item Identificação dos usuários similares ao usuário alvo e seus respectivos níveis de similaridade;
\item Ordenação dos níveis de similaridade para determinação de um novo grupo de usuários que possuem os maiores níveis; 
\item Verificação das avaliações dadas pelos usuários do novo grupo e definição dos itens mais relevantes a serem recomendados.
\end{itemize} 

\section{Estrutura de Dados} \label{sec:estrutura-dados}

A estrutura de dados implementada nesta solução consiste em três classes para mapeamento de cada entidade encontrada no MovieLens, que são \textit{User}, \textit{Movie}, \textit{Rating}, e duas classes para agrupamento das informações e armazenamento das predições, que são \textit{TOMovieLens} e \textit{TOUserRec}. Como pode ser visualizado no diagrama (Figura~\ref{fig:estrutura-dados}), não foi realizado nenhum relacionamento de inclusão ou composição entre as classes.

\begin{figure}[ht]
\centering
\includegraphics[width=1\textwidth]{estrutura-dados.png}
\caption{Estrutura de Dados}
\label{fig:estrutura-dados}
\end{figure}

\section{Algoritmo}\label{sec:algoritmos}
No contexto deste trabalho foi utilizada a abordagem baseada em usuários, utilizando-se da filtragem colaborativa para recomendação de conteúdo (e.g., filmes, páginas, lugares) através do relacionamentos e aproximação entre usuários.
\par
A recomendação de filmes foi implementada utilizando a similaridade entre as avaliações dos usuários, através do \textit{Coeficiente e Correlação de Pearson} (Equação~\ref{eq:pearson}), obtendo assim, a similaridade entre os usuários e possibilitando o agrupamento de usuários com maior similaridade para melhor recomendação.
Após obter a similaridade entre os usuários, foi utilizada a técnica de agrupamento por vizinhos mais próximos (\textit{k-Nearest Neighbor}), onde é escolhido uma quantidade de usuários (k) que possuem maior semelhança, e assim aplicar o algoritmo de predição\ref{eq:predicao} levando em consideração apenas os \textit{k} usuários com maior similaridade.

\begin{itemize}

\item Pearson
\begin{equation}
 \label{eq:pearson}
 pearson(a,b) = \frac{\sum_{p\in P}{(r_{a,p}-\overline{r_a})\times(r_{b,p}-\overline{r_b})}}{\sqrt{\sum_{p\in P}{(r_{a,p}-\overline{r_a})^{2}}}\times\sqrt{\sum_{p\in P}{(r_{b,p}-\overline{r_b})^{2}}}}
\end{equation}
 Onde:
 \begin{itemize}
    \item $a$ é o usuário a ter o item $p$ predito;
    \item $\overline{r_{x}}$ é a média dos \textit{ratings} do usuário $x$;
	\item $P$ é o conjunto de itens avaliados por $a$ e $b$.
	\item $r_{x,p}$ é o \textit{rating} do usuário $x$ para o item $p$;
\end{itemize}

\item Algoritmo de Predição
\begin{equation}
 \label{eq:predicao}
 pred(a, p) = \overline{r_{a}} + \frac{\sum_{b\in N}{sim(a,b) \times (r_{b,p}-\overline{r_b})}}{\sum_{b\in N}{sim(a,b)}}
\end{equation}
\noindent Onde:
\begin{itemize}
 \item $sim(a,b)$ é a função de similaridade entre os usuários $a$ e $b$;
 \item $N$ é o conjunto universo dos usuários.
\end{itemize}
\end{itemize}

\section{Complexidade do Algoritmo} \label{sec:complexidade}

A ordem de complexidade do algoritmo implementado é de \(N^{2}M + NM\) onde \textit{N} é o número de usuários utilizados e \textit{M} o número de items.

\begin{enumerate}
\item FOR \textit{u1} \textbf{(N)} \textit{Calcula similaridade para todos os usuários}
\item \tab{FOR \textit{u2 - u1} \textbf{(N)}}
\item \tab{\tab{FOR \textit{i} \textbf{(M)} \textit{Calcula correlação de Pearson}}}
\item \tab{FOR \textit{i} \textbf{(M)} \textit{Realiza a predição para recomendação}}
\end{enumerate}
\vspace{5 mm}
\lstset{language=Java}
\begin{lstlisting}[frame=single,label=exemplo-java,caption=Exemplo em Java]
for (int i = 0; i < users.size(); i++)
    var neighbors;
    for (int j = i + 1; j < users.size(); j++)
        neighbors = pearson(ratings[i], ratings[j]);
    recommendation(items, neighbors, i);
\end{lstlisting}

\section{Aplicativo} \label{sec:aplicativo}

Neste tópico será descrito como deverá ser utilizado o aplicativo, assim como algumas particularidades e variáveis que foram implementadas neste exercício.

\subsection{Executando a Aplicação}

A execução da aplicação deverá ser realizada via linha de comando, utilizando o \textbf{Java SE 8}. A instrução deverá ser ser a seguinte: \textbf{java -jar tp1.jar Recommender <user> <item> <ratings>}. Onde as entradas de dados deverão estar no padrão a seguir:
\vspace{5 mm}
\begin{description}
\item[User] \hfill \\
id | age | gender | occupation | zip code \\
\textbf{Exemplo:} 1|24|M|technician|85711
\item[Item] \hfill \\
movie id | movie title | release date | video release date | IMDb URL \\
\textbf{Exemplo:} 1|Toy Story (1995)|01-Jan-1995||http://us.imdb.com/M/title-exact?Toy\%20Story\%20(1995)
\item[Ratings] \hfill \\
<user id> <item id> <rating> <timestamp> \\
\textbf{Exemplo:} 1	1	5	874965758

\end{description}
\vspace{5 mm}
\noindent
Caso as entradas de dados estejam incorretas será gerada uma exceção contendo a mensagem \textit{"Input arguments are missing or invalid!"}.

\subsection{Saída de Dados}

A saída de dados será gerada após as recomendações serem realizadas, ao final do processamento das informações será gerado um arquivo \textit{recommendations.txt} contendo os resultados. O arquivo de saída conterá 100 recomendações de itens não avaliados pelo usuário (N), resultando em um total de \textit{N} x 100 linhas, variando de acordo com a entrada de dados contendo os usuários. 

\subsection{Testes de Implementação}

Os testes e a implementação foram realizadas utilizando como entrada de dados os arquivos u.user, u.item, e ua.base disponibilizados no data set da MovieLens 100k\cite{movielens100k}. O tempo de processamento e geração de recomendação neste cenário foi de 185 segundos, utilizando um \textbf{MacBook Pro 13-inc, Mid 2012; 2,5 GHz Intel Core i5; 16 GB 1600 MHz DDR3; OS X 10.9.4 (13E28)}.

\subsection{Parametrização}

O aplicativo possuí alguns métodos e constantes que podem ser ajustados. Elas representam a \textit{Quantidade de Recomendações} que serão geradas, \textit{Alcance da Vizinhança} que será utilizada, \textit{Peso de Significância} utilizado para equilibrar o algoritmo de similaridade quando os usuários possuem poucas avaliações em comum. Os valores padrões utilizados são:

\begin{description}
\item[Quantidade de Recomendações] 100
\item[Alcance da Vizinhança] 20
\item[Peso de Significância] 50
\end{description}
Também foi implementado no \textit{Algoritmo de Correlação de Pearson} (Equação~\ref{eq:pearson}) a possibilidade de utilizar tanto a interseção da avaliação dos itens entre os usuários, como também a união desses itens, utilizando do valor 0 para os itens que não foram avaliados pelo usuário.

\subsection{Implementação}

De forma sucinta, os procedimentos realizados pelo aplicativo são: Leitura das Entradas de Dados e Mapeamento para Classes de Domínio do Aplicativo; Execução do Algoritmo de Similaridade entre os Usuários; Execução do Algoritmo de Predição de Avaliação para cada Item, de acordo com os Usuários Vizinhos; Resultado da Recomendação.

\section{Conclusão} \label{sec:conclusao}

Com esta pequena implementação e teste, pude notar a imensidão e as infinitas possibilidades que podem ser exploradas para recomendações tanto na Filtragem Colaborativa, como também na Baseada em Conteúdo.\\
\indent Durante a implementação deste algoritmo, notei várias possibilidades e variantes que podem existir nos Sistemas de Recomendação, e que devem ser testadas e ajustadas para uma melhor recomendação. Existem várias melhorias a serem realizadas neste sistema, como métricas e avaliações afim de mensurar a eficiência na recomendação.\\
\indent Assim, pretendo em versões futuras, realizar testes mais concretos e ajustes no algoritmo, tal como implementar a similaridade utilizando Cosseno ao invés de Pearson, implementar Recomendação Baseada em Item e realizar medições para comparação, com a finalidade de aprendizado e comportamento desses algoritmos.

\newpage
\bibliographystyle{sbc}
\bibliography{sbc-template}\nocite{*}

\end{document}
